\documentclass[journal]{IEEEtran}
\IEEEoverridecommandlockouts
\usepackage{fancyhdr} %header extension
\usepackage{graphicx}
\usepackage[english]{babel}
\usepackage[utf8]{inputenc}
\usepackage{color}
\usepackage{hyperref}
\usepackage{wrapfig}
\usepackage{array}
\usepackage{multirow}
\usepackage{adjustbox}
\usepackage{nccmath}
\usepackage{subfigure}
\usepackage{amsfonts,latexsym}
\usepackage{enumerate}
\usepackage{booktabs}
\usepackage{float}
\usepackage{threeparttable}
\usepackage{array,colortbl}
\usepackage{ifpdf}
\usepackage{rotating}
\usepackage{cite}
\usepackage{stfloats}
\usepackage{url}
\usepackage{listings}
\newcolumntype{P}[1]{>{\centering\arraybackslash}p{#1}} 
\newcommand{\tabitem}{~~\llap{\textbullet}~~}
\newcommand{\ctt}{\centering\scriptsize\textbf} 
\newcommand{\dtt}{\scriptsize\textbf} 
\renewcommand\IEEEkeywordsname{Keywords}
\hyphenation{} 
\graphicspath{ {Figs/} }

\newcommand{\MYhead}{\smash{\scriptsize
\hfil\parbox[t][\height][t]{\textwidth}{\centering
\begin{picture}(0,0)\put(0,-10){\includegraphics[width=12mm]{iith_logo.png}} \end{picture} \hspace{8.7cm}
FWC\hspace{6.8cm} ASSIGNMENT-1\\
\hspace{7cm} FUTURE WIRELESS COMMUNICATION \hspace{5.5cm} AUG-22\\
\underline{\hspace{ \textwidth}}}\hfil\hbox{}}}
\makeatletter
% normal pages
\def\ps@headings{%
\def\@oddhead{\MYhead}%
\def\@evenhead{\MYhead}}%
% title page
\def\ps@IEEEtitlepagestyle{%
\def\@oddhead{\MYhead}%
\def\@evenhead{\MYhead}}%
\makeatother
% make changes take effect
\pagestyle{headings}
% adjust as needed
\addtolength{\footskip}{0\baselineskip}
\addtolength{\textheight}{-1\baselineskip}

\begin{document}
\title{IMPLEMENTATION OF BOOLEAN LOGIC}
\author{VELICHARLA GOKUL KUMAR\\
				\textit{velicharlagokulkumar@gmail.com}\\
\thanks{}} 
\maketitle
\begin{abstract}
xyz
\end{abstract}

\begin{IEEEkeywords}
IEEE, plantilla, \LaTeX, ecuaciones, 
\end{IEEEkeywords}

\section{Introduction}
\IEEEPARstart{o}{nce}
.
\
.
\\
\\
\\
\\
\\
\\
\\
\\ 
\\ 
\section{concept,Methodology}	
\
\
\
\
\
\
\
\
\
\
\
\
\subsection{Definition}
\
\
\
\
\subsection{Definition 2}
\
\
\
\
\subsection{XYZ}
z
\\
 \cite{nombre_para_citar} \\
 \cite{kopka} \\
 \cite{link} 

\section{Solution proposal}
.
\\
.
\\
.
\\
\\
\\
\\
\\
\\
\\ 
\\
\subsection{xyz \LaTeX}

\begin{equation}\label{eqID}
I_D=\frac{q N_A n_i^2}{N_D}\left(\frac{\alpha V_{GS}^2}{\mu_o}\right)^3
\end{equation}

\begin{equation}\label{Voeq} 
V_o \approx \int e^XdX
\end{equation}
(\ref{eqID}) (\ref{Voeq}) 
$I_D$  $V_o$ 
.
\\
\\
\begin{gather*}
i=\frac{v}{R}\Longrightarrow i=\frac{5}{500}=10 mA
\end{gather*}
\section{Simulation and Results}

\subsection{Figures en \LaTeX}
.
\\
asdfg
\\
\cite{imagenes}. 

\begin{figure}[H]  
\centering  
%\includegraphics[scale=0.23]{fig}
\caption{And and or gate } 
\end{figure}
.
\\
\\
\\
\\
\\
\\
\\
\\
\ref{lvdt4}  $I_1$ contra $V_1$.
 \emph{DIA} \cite{dia} 
\begin{figure}[H] 
\centering 
\%includegraphics[scale=0.55]{lvdt4}
\caption{Diagrm integrated AD598.} 
\label{diafig}
\end{figure}


\section{Implementation and solution}
 PCB (\emph{Printed Circuit Board})  \emph{scripts} 
\section{Results}
.
\\
\\
\\
\\
\\
\\
\subsection{Tablas en \LaTeX}
\begin{table}[H]
\centering
\caption{Nombre de la tabla}
\label{table1}
\begin{tabular}{c c c}\hline\hline
\textbf{Símbolo} & \textbf{Nombre} & \textbf{Código Latex}\\ \hline
$\alpha$ & alpha & \verb|\alpha| \\
$\mu$ & mu & \verb|\mu|\\
$\beta$ & beta & \verb|\beta|\\
$\Omega$ & Omega & \verb|\Omega| \\\hline \hline
\end{tabular}
\end{table}
.
\\
\\
\\
\\
\\
\\
\section{Conclusion}
.
\\
\\
\\
\\
\\
\ifCLASSOPTIONcaptionsoff
  \newpage
\fi
\begin{thebibliography}{1}
\bibitem{nombre_para_citar}
Inicial1.~Apellido1 and Inicial2.~Apellido2, \emph{Nombre de libro}, \#edición~ed.\hskip 1em plus
  0.5em minus 0.4em\relax Ciudad, País: Editorial, año.
	
\bibitem{kopka}
H.~Kopka and P.~W. Daly, \emph{A Guide to \LaTeX}, 3rd~ed.\hskip 1em plus
  0.5em minus 0.4em\relax Harlow, England: Addison-Wesley, 1999.
	
\bibitem{link}
Overleaf. \url{https://www.overleaf.com/}. Recuperado el 30 de Enero de 2017.

\bibitem{imagenes}
Youtube, canal schaparro. \url{https://youtu.be/IhvF6iY7n5k}. Recuperado el 30 de Enero de 2017.

\bibitem{dia}
Dia Diagram Editor. \url{https://sourceforge.net/projects/dia-installer/}. Recuperado el 30 de Enero de 2017.

\end{thebibliography}
\end{document}




